% !TeX spellcheck = en_GB
% !TeX encoding = UTF-8
% !TeX root = ../report.tex

\chapter{Implementation}
\label{chp:Implementation}

\section{Steering}

As already mentioned in chapter one, we used a compact power steering unit, with a rated torque of around 3 Nm. In order to install the unit, the kart's steering column was cut up, as was the power steering's drive shaft. The column and the drive shaft were then joined together by using press-fit joints. To counter the motor's torque, we fixed it to the steering wheel's metal holder and reinforced it with an additional steel linkage.

\section{Braking}

Braking presented itself to be the technically most difficult aspect in the drive-by-wire system. The kart offered limited space, therefore a very efficient actuator was needed to ensure the Force necessary for an emergency brake. Early testing showed that for a full brake roughly 250-350 N of force was required. However, this testing was done with the kart jacked up, therefore we could not fully rely on these results.
%brake hydraulic complicated not interfering 

Early on we set a few requirements for our brake actuator, namely speed, force and precision. After some research we decided to look further into a high end linear motor by LinMot and neglect the idea of a rotary motor. Our main reasons being that a rotary motor generally was significantly slower and usually required external sensors to be precise enough. 
We also dismissed the intriguing concept of directly interfering with the hydraulic system, for example inserting an electro hydraulic pump into the system. Our main reasoning against this was that it would require a lot of time, effort and would most likely not outperfom a simpler mechanical brake actuator.

A very easy to implement and cost efficient method of braking that we proposed was plugging braking. The idea was to use the kart's electro motors to brake, by reversing the current. While it would not need any additional parts and would only require some programming, we soon realised the braking power generated by the hydraulic brake sufficed for a full brake and power consumption would rise significantly. Because time was running low and we had to decide for a reliable solution, we chose a linear motor over the plugging braking option. If we had more time on our hands and could have done some testing of the required braking power, we might have decided otherwise.

With all other options out of our way, we were now able to focus on the linear motor.
With our battery providing 48 VDC, LinMot's range of motor narrowed down quite a bit. The PS01-48x240F-C was our first idea, as it provided up to 572 N of maximum force. We quickly realized that our limited space would not allow for such a big motor. After a personal consultation with one of LinMot's employees, we decided for a PS01-37x120F-HP-C with a 300 mm runner. Because of the smaller size, we now had various options of where to put the linear motor. Again, we strived for an easy installation. Therefore we determined the best location to be on top of the brake cylinder, just in between the two pedals. Pushing the pedal resulted in a slitghly arcshaped downward motion of the braking lever. This is the point where the motor's runner has been attached to the lever. As the linear motor should not be under radial load, a design had to be realized which allowed the motor to tilt slightly. This helped to reduce the radial load, as the downward motion could be compensated with tilting the motor at an angle.

This notion was put into place by mounting the Linear motor on a metal frame right above the Braking Cylinder. The frame is made of two sheets of metal connected by multiple steel linkages. The motor is held on the frame with two ball-bearings, allowing for a pitch motion and therefore preventing any major radial forces.
The metal sheets were cut by a Waterjet cutter.





\section{Throttle}

To control the throttle, we had to access the ACD 4805 motor controller of each electric drive. After a simple start up sequence, we were able to control the velocity of each wheel via CANOpen. For throttling, no actuators had to implemented. The crucial work that went into throttle-by-wire part was mainly communication and will therefore be discussed in the following chapter.

